\chapter{Appendix B: Glossary}

\section{Terms \& Definitions}

\begin{description}
    \item[Adaptive Bitrate Streaming (ABS)] A technique used in streaming multimedia over computer networks. While in the past most video streaming technologies utilized streaming protocols such as RTP with RTSP, today's adaptive streaming technologies are almost exclusively based on HTTP.
    
    \item[API Gateway] An API gateway is a component that sits between clients and microservices. It is responsible for request routing, composition, and protocol translation. It provides each of the application's clients with a custom API.
    
    \item[Canary Deployment] A deployment strategy in which the new version of an application is deployed to a small subset of users before being rolled out to the entire infrastructure. This allows for testing in production with minimal impact.
    
    \item[Circuit Breaker] A design pattern used in software development. It is used to detect failures and encapsulates the logic of preventing a failure from constantly recurring, during maintenance, temporary external system failure or unexpected system difficulties.
    
    \item[Containerization] A form of operating system virtualization, through which applications are run in isolated user spaces called containers, all using the same shared operating system (OS).
    
    \item[Continuous Integration (CI)] The practice of merging all developers' working copies to a shared mainline several times a day.
    
    \item[Continuous Deployment (CD)] A software engineering approach in which software functionalities are delivered frequently through automated deployments.
    
    \item[Database Sharding] A type of database partitioning that separates very large databases the into smaller, faster, more easily managed parts called data shards.
    
    \item[DevOps] A set of practices that combines software development (Dev) and IT operations (Ops). It aims to shorten the systems development life cycle and provide continuous delivery with high software quality.
    
    \item[Docker] A set of platform as a service (PaaS) products that use OS-level virtualization to deliver software in packages called containers.
    
    \item[Elasticsearch] A search engine based on the Lucene library. It provides a distributed, multitenant-capable full-text search engine with an HTTP web interface and schema-free JSON documents.
    
    \item[Event-Driven Architecture] A software architecture paradigm promoting the production, detection, consumption of, and reaction to events.
    
    \item[Federated Identity] A means of linking a person's electronic identity and attributes, stored across multiple distinct identity management systems.
    
    \item[Grafana] A multi-platform open source analytics and interactive visualization web application. It provides charts, graphs, and alerts for the web when connected to supported data sources.
    
    \item[GraphQL] An open-source data query and manipulation language for APIs, and a runtime for fulfilling queries with existing data.
    
    \item[Horizontal Pod Autoscaler (HPA)] An API resource in Kubernetes that automatically scales the number of pods in a replication controller, deployment, replica set, or stateful set based on observed CPU utilization.
    
    \item[Idempotency] A property of certain operations in mathematics and computer science whereby they can be applied multiple times without changing the result beyond the initial application.
    
    \item[Ingress Controllers] A specialized load balancer for Kubernetes (and other container orchestrators) that manages external access to the services in a cluster, typically HTTP.
    
    \item[JWT (JSON Web Token)] An open standard (RFC 7519) that defines a compact and self-contained way for securely transmitting information between parties as a JSON object.
    
    \item[Kubernetes (K8s)] An open-source container-orchestration system for automating computer application deployment, scaling, and management.
    
    \item[Microservices] A software development technique—a variant of the service-oriented architecture (SOA) architectural style that structures an application as a collection of loosely coupled services.
    
    \item[Monolith] A software application in which different components are combined into a single program from a single platform.
    
    \item[mTLS (Mutual TLS)] A method for two-way authentication between standard and client-server connections. It ensures that traffic is secure and trusted in both directions.
    
    \item[OAuth 2.0] An open standard for access delegation, commonly used as a way for Internet users to grant websites or applications access to their information on other websites but without giving them the passwords.
    
    \item[Prometheus] A free software application used for event monitoring and alerting. It records real-time metrics in a time series database built using a HTTP pull model, with flexible queries and real-time alerting.
    
    \item[RAG (Retrieval-Augmented Generation)] A technique for enhancing the accuracy and reliability of generative AI models with facts fetched from external sources.
    
    \item[RBAC (Role-Based Access Control)] A policy-neutral access-control mechanism defined around roles and privileges.
    
    \item[Redis] An in-memory data structure store, used as a distributed, in-memory key–value database, cache and message broker, with optional durability.
    
    \item[REST (Representational State Transfer)] A software architectural style that defines a set of constraints to be used for creating Web services.
    
    \item[Single Sign-On (SSO)] an authentication scheme that allows a user to log in with a single ID and password to any of several related, yet independent, software systems.
    
    \item[Terraform] An open-source infrastructure as code software tool created by HashiCorp. Users define and provide data center infrastructure using a declarative configuration language known as HashiCorp Configuration Language (HCL).
    
    \item[Vector Embedding] A numerical representation of a word, sentence, or document that captures its semantic meaning.
    
    \item[Zero Downtime Deployment] A deployment method where the website or application is never down or in an unstable state during the deployment process.
\end{description}
