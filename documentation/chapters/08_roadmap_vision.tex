\chapter{Roadmap \& Vision}

\section{Phase 1: The "Mobile First" Pivot (Q2 2026)}
\begin{featurebox}[The "Subway Mode" Initiative]
Students are increasingly moving away from laptops and desktops. Our analytics show that 40\% of logins now occur on mobile devices, often via 4G/5G networks. To support this, we will launch native React Native apps for iOS and Android.
\textbf{Key Feature:} The app will intelligently pre-download the next 3 lessons based on the student's current progress. This means that if a student watches "Lesson 1: Variables" while on Wi-Fi at home, the app will secretly download "Lesson 2: Loops" and "Lesson 3: Functions". When the student gets on the subway and loses signal, they can continue learning seamlessly without buffering.
\end{featurebox}

We are also investigating "Lite Mode" for developing nations. This involves stripping out heavy JS frameworks and serving server-rendered HTML with highly compressed WebP images to support students on 2G connections.

\section{Phase 2: The "Global Classroom" (2027)}
Language should not be a barrier to education. Currently, 90\% of high-quality technical content is in English.
\textbf{Real-time Translation:} We plan to use multimodal LLMs to dub video lectures into Spanish, French, Hindi, and Mandarin in real-time. This is not just subtitles; we aim to use "Voice Cloning" technology to make it sound like the original professor is speaking the target language, preserving their intonation and emphasis.
\textbf{Localized Pricing:} We will implement dynamic Purchasing Power Parity (PPP) adjustments. A course that costs \$50 in the US might cost \$5 in India, ensuring equality of access while maximizing global revenue.

\section{Phase 3: Deep Personalization (2028)}
A "Knowledge Graph" that spans across institutions.
Currently, if a student learns "Linear Algebra" in a generic Math course, their credential in the Engineering course doesn't "know" they've done it.
We aim to build a \textbf{Universal Learning Ledger}. This blockchain-backed ledger will store atomic units of competency. If a student proves they know "React Hooks" in one course, they automatically get credit for it in every other course on the platform. This moves us away from "Degrees" and towards "Skill Stacks".

\section{Conclusion}
Shikshak is an ambitious attempt to codify the art of teaching. It is not just code; it is a commitment to the future of human potential. By combining the empathy of human tutors with the infinite patience and scale of Artificial Intelligence, we can build a world where anyone, anywhere, can learn anything.
