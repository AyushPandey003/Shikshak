\chapter{Detailed Microservices Breakdown}

\section{API Gateway}
The gateway is the single point of entry. It handles:
\begin{itemize}
    \item \textbf{SSL Termination}: High-performance decryption.
    \item \textbf{Rate Limiting}: Preventing abuse.
    \item \textbf{Routing}: Directing \texttt{/api/courses} to the Course Service.
\end{itemize}

\section{RAG Service (The "Brain")}
This service powers the AI features. It uses a Retrieval-Augmented Generation pipeline.

\begin{figure}[H]
\centering
\begin{tikzpicture}[
    node distance=2cm,
    process/.style={
        rectangle,
        draw=shikscolPrimary,
        fill=shikscolLight,
        rounded corners,
        minimum height=1.5cm,
        minimum width=3cm,
        align=center,
        font=\sffamily\bfseries
    },
    data/.style={
        cylinder,
        shape border rotate=90,
        aspect=0.25,
        draw=shikscolDark,
        fill=white,
        minimum height=1cm,
        minimum width=1cm,
        align=center
    }
]

\node[process] (query) {User Query};
\node[process, right=of query] (embed) {Embedding\\Model};
\node[data, right=of embed] (vector) {Vector\\Index};
\node[process, below=of vector] (retrieve) {Context\\Retrieval};
\node[process, left=of retrieve] (llm) {LLM\\(GPT-4o)};
\node[process, left=of llm] (answer) {Final Answer};

\draw[->, thick] (query) -- (embed);
\draw[->, thick] (embed) -- (vector);
\draw[->, thick, dashed] (vector) -- (retrieve);
\draw[->, thick] (retrieve) -- (llm);
\draw[->, thick] (llm) -- (answer);
\draw[->, thick, bend right] (answer) to (query);

\end{tikzpicture}
\caption{Artificial Intelligence Query Logic}
\end{figure}

\section{Course Management Service}
Handles the "nouns" of the system:
\begin{itemize}
    \item Courses
    \item Modules
    \item Lessons
    \item Metadata (Tags, Difficulty Levels)
\end{itemize}
It is built on Node.js and MongoDB to handle flexible schema requirements (e.g., some lessons are videos, some are quizzes).
